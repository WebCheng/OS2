\documentclass[english,10pt]{article}
\usepackage[doublespacing]{setspace}
\usepackage{geometry}
\geometry{legalpaper, margin=.75in}
\usepackage[utf8]{inputenc} 
\usepackage{babel}
\usepackage{amsmath}
\usepackage{graphicx}
\usepackage{fancyhdr} 
\newenvironment{singlecenter}{\begin{singlespace}\begin{center}}{\end{center}\end{singlespace}}
\renewcommand{\headrulewidth}{0pt}
\setlength{\headheight}{12pt} 

\begin{document}
%IEEE format help me to check it
%Ellen 94 write "title,class,term"
\title{\bf(Title) 256 characters maximum, including spaces}
%Ellen 94 write "name"
\author{(Authors) Taro NANKYOKU\textsuperscript{1}and Firstname LASTNAME\textsuperscript{2{*}}}

\maketitle 

\begin{singlespace}
% Ellen 94 write abstraction below
{\bf Abstract.} {\it500 words maximum (4000 character maximum).} The abstract should contain a concise summary and mention acquired and prepared data sets as well as possibilities for reusing those data sets. However, no new scientific findings should be presented here. (No citations may be included in the abstract.)
% Ellen 94 write abstraction above
 
%Don't write this part 
\begin{center}
{\bf 1. write-up of your concurrency solution}
\end{center}

% Ellen write "An explanation of each and every flag in the listed qemu command-line"
% Ref Link 
% https://qemu.weilnetz.de/w64/2012/2012-12-04/qemu-doc.html#index-g_t_002denable_002dkvm-92
%qemu-system-i386 -gdb tcp::???? -S -nographic -kernel bzImage-qemux86.bin -drive file=core-image-lsb-sdk-qemux86.ext4,if=virtio -enable-kvm -net none -usb -localtime --no-reboot --append "root=/dev/vda rw console=ttyS0 debug".
\begin{center}
{\bf 2. Flag in the listed qemu command-line}
\end{center}
-gdb ->Wait for gdb connection on device dev\\
tcp::->QEMU will wait for a client socket application to connect to the port before continuing, unless the nowait option was specified. \\
-s->Shorthand for -gdb tcp::1234, i.e. open a gdbserver on TCP port 1234\\
-nographic->totally disable graphical output so that QEMU is a simple command line application\\
-kernel\\
-drive\\


 In this section, authors must explain the background of research that served as the basis for prepared data and explain the composition of that research with citations where necessary. They must also mention their motivation and purpose for preparing the data as well as the data’s value.
  

\begin{center}
{\bf 3.1 What do you think the main point of this assignment is?}
\end{center}


\begin{center}
{\bf 3.2 How did you personally approach the problem? Design decisions, algorithm, etc?}
\end{center}


\begin{center}
{\bf 3.3 How did you ensure your solution was correct? Testing details, for instance.?}
\end{center}

\begin{center}
{\bf 3.4 What did you learn?}
\end{center}


\begin{center}
{\bf 2.Version control log}
\end{center}
\begin{tabular}{rrrrr}
one & two & three & four & five\\\hline
1.23 & 3.45 & 5.00 & 1.21 & 3.41 \\
1.23 & 3.45 & 5.00 & 1.21 & 3.42 \\
1.23 & 3.45 & 5.00 & 1.21 & 3.43 \\
\end{tabular}  

\begin{center}
{\bf 2. Work log. What was done when? Be detailed.}
\end{center}
4/5\\
VM Boot Question\\
4/6\\
VM Boot success\\
4/7\\
Git for the VM code!!\\
Overleaf for the Latex\\
Git for the c concurrency solution\\

\end{singlespace}
 
\end{document}




